\documentclass[11pt, a4paper]{article}
\usepackage[margin=1in]{geometry}

\usepackage{lmodern}
\fontfamily{lmdh}\selectfont

\usepackage{amsfonts} 

\usepackage{graphicx}
\graphicspath{{../}}


\usepackage{titlesec}
\titleformat*{\section}{\Large\bfseries}
\titleformat*{\subsection}{\large\bfseries}

\usepackage{caption}

\usepackage{subcaption}

\usepackage[style=authoryear-ibid,backend=biber]{biblatex}
\renewcommand*{\nameyeardelim}{\addcomma\space}
\addbibresource{ag_bib.bib}

\usepackage{verbatim}

\usepackage[section]{placeins}

\title{\LARGE\bfseries ZZCA6510 3: Prescriptive Decision Analysis}
\author{\Large Amanjit Gill}
\date{\normalsize \today}

\begin{document}
    
    \maketitle

    \thispagestyle{empty}

    \newpage

    \addtocounter{page}{-1}

    \section{Product Composition}

    Peanut for Life, a food manufacturer, seeks to produce a chanachur snack product comprising three mixtures: puffed rice, nuts and cereal. Each ingredient mixture has a different cost per kilogram (see Table \ref{costs}); therefore, Peanut for Life is aiming to minimise the total cost of each package of chanachur by formulating a linear programming model. 
    
    \begin{table}[!ht]
        \centering
        \caption{Cost of each ingredient.}
        \begin{tabular}{|l|l|}
            \hline
            Ingredient & Cost (\$/kg) \\ \hline
            Puffed rice mix & 0.35 \\ \hline
            Nut mix & 0.50 \\ \hline
            Cereal mix & 0.20 \\ \hline          
        \end{tabular}
        \label{costs}
    \end{table}
    
    
    If $x_{1}$, $x_{2}$ and $x_{3}$ represent the weight (in kilograms) of puffed rice, nuts and cereal respectively, then the total cost of one package of chanachur would be:

    \begin{equation}
        total\_cost = 0.35x_1 + 0.5x_2 + 0.2x_3
        \label{obj_func1}
    \end{equation}
    
    The objective is to \textit{minimise} this quantity. However, there is not just the cost to consider. Peanut for Life must also ensure the product is commercially viable; it can do this by composing the product in such a way that it is both nutritionally balanced and attractive to consumers. To this end, the following constraints on the final composition are given:

    \begin{itemize}
        \item The chanachur package must hold between 3 and 4 cups of product.
        \item One package cannot contain more than 1000 calories in food energy.
        \item One package cannot contain more than 25 grams of fat.
        \item At least 20\% of the product volume must comprise puffed rice mixture.
        \item No more than 15\% of the product weight may comprise nuts.
    \end{itemize}

    Using the nutritional information provided in Table \ref{nutrients}, these constraints may be modelled mathematically as inequalities \ref{first_constraint1} to \ref{last_constraint1}, presented in \textit{canonical form}.

    \begin{table}[!ht]
        \centering
        \caption{Nutritional information per kg.}
        \begin{tabular}{|l|l|l|l|}
            \hline
            Ingredient      & Volume (cups)     & Fat (g)   & Calories  \\ \hline
            Puffed rice mix & 0.25              & 0         & 150       \\ \hline
            Nut mix         & 0.375             & 10        & 400       \\ \hline
            Cereal mix      & 1.0               & 1         & 50        \\ \hline          
        \end{tabular}
        \label{nutrients}
    \end{table}

    \begin{equation}
        -0.25x_1 - 0.375x_2 - x_3 \leq -3
        \label{first_constraint1}
    \end{equation}

    \begin{equation}
        0.25x_1 + 0.375x_2 + 1x_3 \leq 4
    \end{equation}
    
    \begin{equation}
        150x_1 + 400x_2 + 50x_3 \leq 1000
    \end{equation}

    \begin{equation}
        10x_2 + 1x_3 \leq 25
    \end{equation}

    \begin{equation}
        -0.2x_1 + 0.075x_2 + 0.2x_3 \leq 0
    \end{equation}

    \begin{equation}
        -0.15x_1 + 0.85x_2 - 0.15x_3 \leq 0
        \label{last_constraint1}
    \end{equation}

    One additional constraint is required; that is, a nonnegativity constraint for the weight of each ingredient:

    \begin{equation}
        x_i \geq 0 \textrm{, where } i \in \{1, 2, 3\}
        \label{nonneg_constraint1}
    \end{equation}

    Equation \ref{obj_func1}, together with inequations \ref{first_constraint1} to \ref{nonneg_constraint1}, completely define the linear programming model for optimising the composition of the new chanachur product. However, when it is solved using the Simplex algorithm, the optimal solution does not include any nuts at all. This is likely due to the high fat and energy content of the nut mixture (as shown in Table \ref{nutrients}) encouraging the solver to favour the less nutritionally dense puffed rice and cereal mixtures.

    From a marketability standpoint, this is an unacceptable result, as nuts cannot be excluded from a commercially viable chanachur product. For this reason, a minimum amount of nut mixture must be enforced through an additional constraint. A cursory examination of popular chanachur recipes reveals that most small batches contain between 0.5 and 1.0 cups of nuts; approximately 75-113 grams. Therefore, for the present exercise, a nominal minimum of 100 grams has been chosen, and represented by inequation \ref{nut_min} in \textit{canonical form}.

    \begin{equation}
        -x_2 \leq -0.1
        \label{nut_min}
    \end{equation}

    When the model is solved with this new constraint, a more suitable composition is achieved. This is given in Table \ref{results1}. The corresponding minimal cost per package is \$1.36.

    \begin{table}[!ht]
        \centering
        \caption{Optimal composition of chanachur.}
        \begin{tabular}{|l|l|}
            \hline
            Ingredient                  & Optimal weight (kg)   \\ \hline
            Puffed rice mix ($x_1$)     & 2.4                   \\ \hline
            Nut mix ($x_2$)             & 0.1                   \\ \hline
            Cereal mix ($x_3$)          & 2.363                 \\ \hline          
        \end{tabular}
        \label{results1}
    \end{table}

    Appendix XXXX contains the model construction in Microsoft Excel.

    \newpage

    \section{Staff Scheduling}

    Chris Stokes, the rostering manager at Orient Computer Manufacturer, has been tasked with developing a staff schedule that minimises the total weekly cost of salary payments. Employees at Orient work in weekly shifts that include two days off, and their weekly salary depends on which days they are rostered on, as shown in Table \ref{shifts}. In addition, Mr Stokes has estimated the number of workers required on the factory floor every day of the week; this is shown in Table \ref{workers_needed}. 

    \begin{table}[!ht]
        \centering
        \caption{Weekly salary per shift.}
        \begin{tabular}{|l|l|l|}
            \hline
            Shift   & Days off      & Salary (\$)  \\ \hline
            1       & Sun and Mon   & 900          \\ \hline
            2       & Mon and Tue   & 850          \\ \hline
            3       & Tues and Wed  & 920          \\ \hline
            4       & Wed and Thu   & 860          \\ \hline
            5       & Thu and Fri   & 780          \\ \hline
            6       & Fri and Sat   & 910          \\ \hline
            7       & Sat and Sun   & 850          \\ \hline
        \end{tabular}
        \label{shifts}
    \end{table}

    \begin{table}[!ht]
        \centering
        \caption{Workers required per day.}
        \begin{tabular}{|l|l|}
            \hline
            Day     & No. workers required  \\ \hline
            Sun     & 18                    \\ \hline
            Mon     & 13                    \\ \hline
            Tue     & 15                    \\ \hline
            Wed     & 18                    \\ \hline
            Thu     & 21                    \\ \hline
            Fri     & 18                    \\ \hline
            Sat     & 21                    \\ \hline     
        \end{tabular}
        \label{workers_needed}
    \end{table}

    Mr Stokes sees that a linear programming approach is ideally suited to this scenario. To this end, if $x_{1}$ represents the number of employees assigned to shift 1, $x_{2}$ represents the number of employees assigned to shift 2, and so on, then the total salary cost for one week would be:

    \begin{equation}
        total\_salary = 900x_1 + 850x_2 + 920x_3 + 860x_4 + 780x_5 + 910x_6 + 850x_7
        \label{obj_func2}
    \end{equation}

    The objective is to \textit{minimise} this cost while ensuring that the staff requirement for every day is met. This can be done by representing the number of workers required every day, given in Table \ref{workers_needed}, as a set of constraints. These manifest themselves as inequations \ref{first_constraint2} to \ref{last_constraint2}.

    \begin{equation}
        x_2 + x_3 + x_4 + x_5 + x_6 \geq 18   
        \label{first_constraint2}     
    \end{equation}

    \begin{equation}
        x_3 + x_4 + x_5 + x_6 + x_7 \geq 13
    \end{equation}

    \begin{equation}
        x_1 + x_4 + x_5 + x_6 + x_7 \geq 15 
    \end{equation}

    \begin{equation}
        x_1 + x_2 + x_5 + x_6 + x_7 \geq 18
    \end{equation}

    \begin{equation}
        x_1 + x_2 + x_3 + x_6 + x_7 \geq 21
    \end{equation}

    \begin{equation}
        x_1 + x_2 + x_3 + x_4 + x_7 \geq 18
    \end{equation}

    \begin{equation}
        x_1 + x_2 + x_3 + x_4 + x_5 \geq 21
        \label{last_constraint2}
    \end{equation}

    There are additional constraints; namely, that each five-day shift must be assigned at least one worker, and that the number of workers assigned to each shift is a nonnegative integer. While part-time assignments are possible in other scenarios, Orient states that its employees are entitled to two days off per week, implying that its entire staff body works full time i.e. five days per week.

    The additional constraints are represented by inequations \ref{atleast_one} and \ref{int_constraint}. It is also conventional to explicitly require nonnegativity (see inequation \ref{nonneg_constraint2}) but inequation \ref{atleast_one} is sufficiently restrictive.

    \begin{equation}
        x_i \geq 1 \textrm{, where } i \in \{1, 2, .., 7\}
        \label{atleast_one}
    \end{equation}

    \begin{equation}
        x_i \in \mathbb{Z} \textrm{, where } i \in \{1, 2, .., 7\}
        \label{int_constraint}
    \end{equation}

    \begin{equation}
        x_i \geq 0 \textrm{, where } i \in \{1, 2, .., 7\}
        \label{nonneg_constraint2}
    \end{equation}

    The objective function, given by equation \ref{obj_func2}, together with these constraints form a complete mathematical model suitable for linear programming. When this model is solved, an optimal number of employees working each five-day shift is obtained (see Table \ref{results2}). The corresponding minimal salary cost is \$22,410. 

    \begin{table}[!ht]
        \centering
        \caption{Optimal shift allocation.}
        \begin{tabular}{|l|l|}
            \hline
            Shift   & Optimal allocation    \\ \hline
            1       & 5                     \\ \hline
            2       & 8                     \\ \hline
            3       & 3                     \\ \hline
            4       & 1                     \\ \hline
            5       & 4                     \\ \hline
            6       & 2                     \\ \hline
            7       & 3                     \\ \hline
        \end{tabular}
        \label{results2}
    \end{table}

    See Appendix XXXX for the model construction in Microsoft Excel.

    \newpage

    \section{Independent Analysis}

    \subsection{Executive Summary}



    \subsection{Introduction}

    The purpose of this analysis is to formulate a new method by which funding can be allocated to public schools, in such a manner that emphasises support for students who underperform academically due to disability or social factors. 
    
    This study specifically focuses on public secondary schools in a single municipality, the City of Casey in Victoria. The scope has been such limited for the following reasons:

    \begin{itemize}
        \item Because this study aims to function as a proof of concept, it is desirable to limit its scope to a small subset of schools, as this allows the efficacy of the proposed methodology to be demonstrated without the complexity inherent in larger scale modelling.
        \item Eliminating primary schools from the analysis removes the possibility that uncontrolled differences between the primary and secondary school systems might affect the results and conclusions.
        \item Confining the analysis to a single municipality, the City of Casey, eliminates any possible influence of demographic differences between municipalities.  
    \end{itemize}

    Students who require extra support at school are sometimes provided access to Education Support (ES) workers. These roles are filled by paraprofessionals who do not perform a teaching role and who are usually not degree-qualified educators. Because of this, they are counted among ``non-teaching staff''. Publicly available data from ACARA \parencite{acara_profiles} show that for non-government schools, the median ratio of enrolled students to non-teaching staff is much lower than for government schools. Table \ref{es_staff_ratios} shows these ratios for the City of Casey.

    \begin{table}[!ht]
        \centering
        \caption{Median ratio of students to non-teaching staff in Casey.}
        \begin{tabular}{|l|l|}
            \hline
            Sector          & Median Ratio  \\ \hline
            non-government  & 21            \\ \hline
            government      & 34            \\ \hline
        \end{tabular}
        \label{es_staff_ratios}
    \end{table}

    These figures suggest that in the public education sector, funding and provision of non-teaching staff, including ES workers, may be falling short of demand, and that students at non-government schools may be better-serviced than those in government schools. In order to understand why this may be happening, it is instructive to examine how schools in Victoria - and Australia generally - are funded.

    State governments are largely responsible for funding public schools, although the federal government also contributes a smaller amount. Each year, the federal government calculates the SRS (Schooling Resource Standard), which is the amount it costs to educate a child for one year \parencite{srs_background}. In 2020, the base SRS amount was \$14,761 for secondary students \parencite{srs_2020}.

    After adjusting for school-specific ``loadings'' \parencite{srs_2020}, the federal government provides 20\% of the SRS amount to government schools, and 80\% of it to non-government schools. The Victorian government is thus responsible for providing 80\% of the SRS amount to government schools, and 20\% to non-government schools. 

    In practice, the Victorian government divides its funding burden into two categories: the core student learning allocation (that is, the basic amount allocated to every student) and equity funding \parencite{srp_vic}. The amount of equity funding a school receives depends on the specific demographic and academic profile of the school, and takes into account the academic needs of individual students. One type of equity funding comes from the PSD (Program for Students with Disabilities). To attract PSD funding to a school, a student must meet specific disability criteria, and the allocated amount varies with the severity of the disability \parencite{psd_guidelines}.

    While this is a sound model in principle, it is problematic for the following reasons:

    \begin{itemize}
        \item Students with ADHD (Attention Deficit Hyperactivity Disorder) do not qualify for extra funding \parencite{psd_guidelines}. According to \Citeauthor{adhd} (\citedate{adhd}), school functioning is strongly impaired by ADHD, a condition common to 5.9\% of youths. This means that no children with ADHD, who do not meet other criteria under the PSD guidelines, receive any support at school, despite conclusive evidence of severe impacts to educational attainment.
        \item Research such as that by \Citeauthor{aces} (\citedate{aces}) shows that children with a high number of ACEs (adverse childhood experiences) are more likely to suffer poor school attendance and poor academic achievement. Unless such children exhibit behavioural problems severe enough to meet PSD funding criteria, their underachievement will go unnoticed and unaddressed.
        \item Even children whose diagnoses meet the PSD criteria are often unable to attract sufficient funding. At one school within the City of Casey, a student whose executive functioning was severely impaired by Autism Spectrum Disorder was allocated an ES worker for only one mathematics lesson per week, meaning he was unable to make any progress for the remaining four lessons every week. The lack of access to support for this student, despite a confirmed urgent need, indicates that current funding provisions are either inadequate or misapplied.
    \end{itemize}

    These observations about the shortcomings of the Victorian government's PSD are the foundation of the present study. They make the case that a more inclusive, targeted, funding model is required; one that captures students who would otherwise fail to meet funding criteria, and one that provides support that is more proportionate to the needs of students who are severely impacted by disabilities.

    School-specific data required for this work have been obtained from government websites, while information about state and federal funding models has been obtained from the extensive guidelines and documentation published by government education departments. This context-specific information has been supplemented by examining research articles about educational practices (especially in the context of students with higher needs) and also about the applicability of linear programming in funding allocation scenarios.

    In collecting data for this analysis, a number of limitations have revealed themselves:

    \begin{itemize}
        \item There is no publicly available data on what proportion of students are underperforming at schools. Data from standardised testing reveal the percentage of students at every school whose two-year progress is ``above average'', but there is no information on the percentage of students whose progress is below average or poor \parencite{naplan}. Because of this, it is difficult to estimate the number of children at a school who might need support but who fall outside the strict PSD criteria. 
        \item A small number of secondary schools in Casey service both primary and secondary year levels \parencite{casey_schools}. There is no readily accessible data on what proportion of these schools' total enrolment comprises secondary students. This means that current funding estimates for these schools are likely to be inaccurate, because the federal government's SRS base amount is different for primary and secondary students \parencite{srs_2020}.
        \item A small number of secondary schools in the catchment areas for this study have only recently opened, so there is no NAPLAN (National Assessment Program – Literacy and Numeracy) data available for them. In addition, a small number schools do not offer VCE (Victorian Certificate of Education) classes, meaning there is no VCE performance data for them.
    \end{itemize}

    These limitations have been mitigated as described in Section \ref{problem_identification}, but will inevitably impact upon the accuracy of the figures. Nevertheless, the results of the analysis suggest that linear programming is a valid, efficient, tool for allocation of funding in narrow contexts such as disability support, and that significant cost savings are possible if education funding is applied in a more targeted way.  

    \subsection{Problem Identification} \label{problem_identification}




    \subsection{Solution Approach}



    \subsection{Results and Discussion}



    \subsection{Conclusion}



    \newpage

    \printbibliography 



\end{document}


