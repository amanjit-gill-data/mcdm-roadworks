\documentclass[11pt, a4paper]{article}
\usepackage[margin=1in]{geometry}

\usepackage{lmodern}
\fontfamily{lmdh}\selectfont

\usepackage{graphicx}
\graphicspath{{../}}

\usepackage{titlesec}
\titleformat*{\section}{\large\bfseries}

\usepackage{caption}

\usepackage{subcaption}

\usepackage[style=authoryear-ibid,backend=biber]{biblatex}
\renewcommand*{\nameyeardelim}{\addcomma\space}
\addbibresource{ag_bib.bib}

\usepackage{verbatim}

\usepackage[section]{placeins}

\title{\Large\bfseries ZZCA6510 3: Quantitative Decision Analysis}
\author{\large Amanjit Gill}
\date{\small \today}

\begin{document}
    
    \maketitle

    \newpage

    \section{Linear Programming}

    Peanut for Life, a food manufacturer, seeks to produce a chanachur snack product comprising three mixtures: puffed rice, nuts and cereal. Each ingredient mixture has a different cost per kilogram (see Table \ref{t1}); therefore, Peanut for Life is aiming to minimise the total cost of each package of chanachur by formulating a linear programming model. 
    
    \begin{table}[!ht]
        \centering
        \begin{tabular}{|l|l|}
            \hline
            ~ & Cost (\$/kg) \\ \hline
            Puffed rice mix& 0.35 \\ \hline
            Nut mix & 0.50 \\ \hline
            Cereal mix & 0.20 \\ \hline          
        \end{tabular}
        \caption{Cost of each ingredient}
        \label{t1}
    \end{table}
    
    
    If $x_{1}$, $x_{2}$ and $x_{3}$ represent the weight (in kilograms) of puffed rice, nuts and cereal respectively, then the total cost of one package of chanachur would be:

    $$ 0.35x_1 + 0.5x_2 + 0.2x_3 $$

    However, there is not just the cost to consider. Peanut for Life must also ensure the product is commercially viable; it can do this by composing the product in such a way that it is both nutritionally balanced and attractive to consumers. To this end, the following constraints on the final composition are given:

    \begin{itemize}
        \item The chanachur package must hold between 3 and 4 cups of product.
        \item One package cannot contain more than 1000 calories in food energy.
        \item One package cannot contain more than 25 grams of fat.
        \item At least 20\% of the product volume must comprise puffed rice mixture.
        \item No more than 15\% of the product weight may comprise nuts.
    \end{itemize}

    Using the nutritional information provided in Table \ref{t2}, these constraints may be modelled mathematically as the following inequalities, presented in \textit{canonical form}:

    $$-0.25x_1 - 0.375x_2 - x_3 <= -3$$
    $$0.25x_1 + 0.375x_2 + 1x_3 <= 4$$
    $$150x_1 + 400x_2 + 50x_3 <= 1000$$
    $$10x_2 + 1x_3 <= 25$$
    $$-0.2x_1 + 0.075x_2 + 0.2x_3 <= 0$$
    $$-0.15x_1 + 0.85x_2 - 0.15x_3 <= 0$$

    


\end{document}


